


%% Дополнительные команды для удобства работы с проектом

%% Вставляет 
\newcommand{\draft}[2] {
	\begin{quotation}
		\begin{center}
			\colorbox{BurntOrange}{\noindent\rule{12cm}{0.4pt}}
		\end{center}
		#1 \cite{#2}
		\\
Источник: \href{run:./materials/#2.pdf}{#2}
		\begin{center}
			\colorbox{BurntOrange}{\rule{12cm}{0.4pt}}
		\end{center}
	\end{quotation}
}

\newcommand{\etc}[2] {
	\begin{quotation}
		#1
		Источник: \href{run:./etc/#2}{#2}
	\end{quotation}
}

\newcommand{\idea}[1] {
	\begin{quotation}
		\begin{center}
			\colorbox{NavyBlue}{\noindent\rule{12cm}{0.4pt}}
		\end{center}
		#1 \\
		\begin{center}
			\colorbox{NavyBlue}{\noindent\rule{12cm}{0.4pt}}
		\end{center}
	\end{quotation}
}

%% Команда для отображения ссылки на список лит-ры и файл данной статьи в materials
\newcommand{\citelink}[1] {
	\cite{#1} \href{run:./materials/#1}{(link)}
}

%% Команда для отображения ссылки на файл в доп.материалах в папке etc
\newcommand{\etclink}[1] {
	\href{run:./etc/#1}{(link)}
}

%% Команда для отображения ссылки на файл в доп.материалах в папке src
\newcommand{\srclink}[1] {
	\href{run:./src/#1}{(link)}
}


%%%%%%%   Исправления для операторов argmin и argmax   %%%%%%%%%%%%%

\DeclareMathOperator*{\argmin}{\arg\!\min}
\DeclareMathOperator*{\argmax}{\arg\!\max}

%%%%%%%%%%%%%%%%%%%%%%%%%%%%%%%%%%%%%%%%%%%%%%%%%%%%%%%%%%%%%%%%%%%%