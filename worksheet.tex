%%% Поля и разметка страницы %%%
\documentclass[a4paper,12pt]{article}
\usepackage{lscape}		% Для включения альбомных страниц

%%% Кодировки и шрифты %%%
\usepackage{cmap}						% Улучшенный поиск русских слов в полученном pdf-файле
\usepackage[T2A]{fontenc}				% Поддержка русских букв
\usepackage[utf8]{inputenc}				% Кодировка utf8
\usepackage[russian,english]{babel}	% Языки: русский, английский
% %\usepackage{pscyr}						% Красивые русские шрифты

%%% Математические пакеты %%%
\usepackage{amsthm,amsfonts,amsmath,amssymb,amscd} % Математические дополнения от AMS

%%% Оформление абзацев %%%
\usepackage{indentfirst} % Красная строка

%%% Поддержка алгоритмов
\usepackage[]{algorithm2e}

%%% Цвета %%%
\usepackage[usenames, dvipsnames]{color}
\usepackage{color}
\usepackage{colortbl}

%%% Таблицы %%%
\usepackage{longtable}					% Длинные таблицы
\usepackage{multirow,makecell,array}	% Улучшенное форматирование таблиц

%%% Общее форматирование
\usepackage[singlelinecheck=off,center]{caption}	% Многострочные подписи
\usepackage{soul}									% Поддержка переносоустойчивых подчёркиваний и зачёркиваний

%%% Библиография %%%
\usepackage{cite} % Красивые ссылки на литературу

%%% Комментарии %%%
\usepackage{verbatim}

%%% Гиперссылки %%%
\usepackage[plainpages=false,pdfpagelabels=false]{hyperref}
\definecolor{linkcolor}{rgb}{0.9,0,0}
\definecolor{citecolor}{rgb}{0,0.6,0}
\definecolor{urlcolor}{rgb}{0,0,1}
\hypersetup{
    colorlinks, linkcolor={linkcolor},
    citecolor={citecolor}, urlcolor={urlcolor}
}

%%% Дополнительные цвета %%%
\usepackage{colortbl}

%%% Подключаем файл с удобными коммандами %%%
%%% \citelink{} - позволяет добавить не просто номер источника, но и гиперссылку на него(при этом файл источника должен лежать в папке с )
%%% \etclink{} -  позволяет добавить ссылку на файл доп.материала (надо указать имя файла из папки etc вместе с расширением)
%%% \srclink{} -  позволяет добавить ссылку на файл исходника (надо указать имя файла из папки src вместе с расширением)



%% Дополнительные команды для удобства работы с проектом

%% Вставляет 
\newcommand{\draft}[2] {
	\begin{quotation}
		\begin{center}
			\colorbox{BurntOrange}{\noindent\rule{12cm}{0.4pt}}
		\end{center}
		#1 \cite{#2}
		\\
Источник: \href{run:./materials/#2.pdf}{#2}
		\begin{center}
			\colorbox{BurntOrange}{\rule{12cm}{0.4pt}}
		\end{center}
	\end{quotation}
}

\newcommand{\etc}[2] {
	\begin{quotation}
		#1
		Источник: \href{run:./etc/#2}{#2}
	\end{quotation}
}

\newcommand{\idea}[1] {
	\begin{quotation}
		\begin{center}
			\colorbox{NavyBlue}{\noindent\rule{12cm}{0.4pt}}
		\end{center}
		#1 \\
		\begin{center}
			\colorbox{NavyBlue}{\noindent\rule{12cm}{0.4pt}}
		\end{center}
	\end{quotation}
}

%% Команда для отображения ссылки на список лит-ры и файл данной статьи в materials
\newcommand{\citelink}[1] {
	\cite{#1} \href{run:./materials/#1}{(link)}
}

%% Команда для отображения ссылки на файл в доп.материалах в папке etc
\newcommand{\etclink}[1] {
	\href{run:./etc/#1}{(link)}
}

%% Команда для отображения ссылки на файл в доп.материалах в папке src
\newcommand{\srclink}[1] {
	\href{run:./src/#1}{(link)}
}


%%%%%%%   Исправления для операторов argmin и argmax   %%%%%%%%%%%%%

\DeclareMathOperator*{\argmin}{\arg\!\min}
\DeclareMathOperator*{\argmax}{\arg\!\max}

%%%%%%%%%%%%%%%%%%%%%%%%%%%%%%%%%%%%%%%%%%%%%%%%%%%%%%%%%%%%%%%%%%%%

%%% Ссылки на локальные файлы %%%
\usepackage{hyperref}

%%%%%%%%%%%%%%%%%%%%%%%%%%%%%%%%%%%%%%%%%%%%%%%%%%%%%%%%%%%%%%%%%%%%


%%% Изображения %%%
\usepackage{graphicx}		% Подключаем пакет работы с графикой
\graphicspath{{images/}}	% Пути к изображениям

%%% Выравнивание и переносы %%%
\sloppy					% Избавляемся от переполнений
\clubpenalty=10000		% Запрещаем разрыв страницы после первой строки абзаца
\widowpenalty=10000		% Запрещаем разрыв страницы после последней строки абзаца

%%% Библиография %%%
\makeatletter
\bibliographystyle{ugost2008}	% Оформляем библиографию в соответствии с ГОСТ 2008
\renewcommand{\@biblabel}[1]{#1.}	% Заменяем библиографию с квадратных скобок на точку:
\makeatother

%%% Колонтитулы %%%
\let\Sectionmark\sectionmark
\def\sectionmark#1{\def\Sectionname{#1}\Sectionmark{#1}}
\makeatletter
\newcommand*{\currentname}{\@currentlabelname}
\renewcommand{\@oddhead}{\it \vbox{\hbox to \textwidth%
    {\hfil  --- Рабочие материалы --- \hfil\strut}\hbox to \textwidth%
    {\today \hfil \thesection~\Sectionname\strut}\hrule}}
\makeatother

%%%%%%%%%%%%%%%%%%%%%%%%%%%%%%%%%%%%%%%%%%%%%%%%%%%%%%%%%%%%%%%%%%%%%%%%%%%%%%%%%%%
\begin{document}

{~}\bigskip
\begin{center}
\Huge{Тема работы}
\end{center}

\section{Введение}

Тема состоит в том, чтобы 
\begin{itemize}
	\item Задача 1
	\item Задача 2
	\item Задача 3
\end{itemize}

На мой взгляд лучший способ освоить эту тему состоит в следующем:
\begin{itemize}
	\item написать реферат по классическим статьям представленным ниже
	\item читать подготовить обзор современных направлений
	\item делать практическую работы \textbf{в следующем формате}
\end{itemize}

\section{Мотивационное письмо}

Для чего нужно развивать данное направление, почему оно перспективно. Какие перспективы это даёт.

\section{Оборудование (опционально)}



\section{Реферат}

Чтобы написать реферат достаточно дать развёрнутый ответ на следующие вопросы:\\
Что такое \textbf{главный термин}?\\
Для каких целей предназначен \textbf{главный термин}?\\
Какова математическая основа \textbf{главный термин}?\\
Каков алгоритм \textbf{главный термин}(описание, схема, плюсы и минусы)?\\
и т.д.\\

\section{Материал}

Основным источником по теме является \textbf{ключевая статья}. Файлы с основными статьями размещаются в папке materials (файлы pdf должны имень имена такие же как ключи bibtex).


\section{План работы}

\begin{enumerate}
	\item сделать реферат по \textbf{ключевая статья} в LaTeX
	\item выполнить первое практическое задание (реализовать простейший случай и провести эксперимент)
	\item дополнить реферат из других источников
	\item подготовить презентацию по этой теме для семинара
	\item расширить и задокументировать программу
	\item провести итоговый эксперимент и задокументировать сделанное
	\item оформить работу
\end{enumerate}


\section{Задания с датами}

Место для конкретных заданий с датами.

\section{Дополнительно}

Далее собраны полезные рекомендации, которые помогут вам успешно справиться.\\

\href{http://compvis.isu.ru/wp-content/uploads/2015/11/PracticalScienceMethodology_1.pdf}{Практические советы по ведению научной работы}

\href{http://compvis.isu.ru/?p=389}{Инструкция для начинающих по работе с LaTeX}\\


\addcontentsline{toc}{chapter}{\bibname}	% Добавляем список литературы в оглавление
\bibliography{biblio}						% Подключаем BibTeX-базы

\end{document}
